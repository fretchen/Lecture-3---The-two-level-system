After the previous discussions of some basic cooking recipes to quantum mechanics in last weeks lectures, we will use them to understand the two-level system. A very detailled discussion can be found in chapter 4 of Ref. \cite{1}. The importance of the two-level system is at least three-fold:
\begin{enumerate}
\item It is the simplest system of quantum mechanics as it spans a Hilbert space of only two states.
\item It is quite ubiquitous in nature and very widely used in atomic physics.
\item The two-level system is another word for the qubit, which is the fundamental building block of the exploding field of quantum compuinformation science
\end{enumerate}
Some of the many examples for two-level systems that can be found in nature:
\begin{itemize}
	\item Spin of the electron: Up vs. down state
	\item Two-level atom with one electron (simplified): Excited vs. ground state
	\item Structures of molecules, e.g., \hyperref[fig:twostate]{NH\textsubscript{3}}
	\item Occupation of mesoscopic capacitors in nanodevices.
	\item Current states in superconducting loops.
	\item Nitrogen-vacancy centers in diamond.
\end{itemize}