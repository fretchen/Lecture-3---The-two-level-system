After the previous discussions of some basic cooking recipes to quantum mechanics in last weeks lectures, we will use them to understand the two-level system. A very detailled discussion can be found in chapter 4 of cf. \cite{1}. The importance of the two-level system is at least two-fold:
\begin{enumerate}
\item It is the simplest system of quantum mechanics as it spans a Hilbert space of only two states.
\item It is quite ubiquitous in nature and very widely used in atomic physics.
\end{enumerate}
Some of the many examples for two-level systems that can be found in nature:
\begin{itemize}
	\item Spin of the electron: Up vs. down state
	\item Two-level atom with one electron (simplified): Excited vs. ground state
	\item Structures of molecules, e.g., \hyperref[fig:twostate]{NH\textsubscript{3}}
	\item Occupation of meso
\end{itemize}
Two-level systems are also used excessively in information technology.
How do we induce dynamics, i.e., population transfer between the two levels $\ket{1}$ and $\ket{2}$? The time evolution is merely
		\begin{align}
			a_1\ket{1}\eexp{-i{E_1 t}/{\hbar}} + a_2\ket{2}\eexp{-i{E_2 t}/{\hbar}},
		\end{align}
		so that the occupation probabilities stay the same. 
		
		So far, we described the system in an eigenbasis of the Hamiltonian.
		If we choose a Hamiltonian which is not diagonal in the basis of interest, we will get a transfer of population between the two states.

\textbf{Example.}
\begin{align}
 H=\frac{\hbar}{2}\Omega_x(\ket{2}\bra{1} + \ket{1}\bra{2}),
\end{align}
where $\Omega_x$ is the Rabi frequency.

\textbf{Note.} Sometimes it can be helpful to work in the Heisenberg picture, where the time evolution is added to the operators describing observables rather than state kets:

\begin{align}
	\hat{A}_H=\eexp{i{\hat{H} t}/{\hbar}} \hat{A}_S \eexp{-i{\hat{H} t}/{\hbar}}
\end{align}
where $\eexp{-i{\hat{H} t}/{\hbar}}$ is a time evolution operator (N.B.: $\hat{H}_S = \hat{H}_H$). The time evolution of $\hat{A}_H$ is:
\begin{align}
 \notag \frac{d}{dt} \hat{A}_H &= \frac{i}{\hbar}\hat{H}\eexp{i{\hat{H}t}/{\hbar}}\hat{A}_S \eexp{-i{\hat{H} t}/{\hbar}}-\frac{i}{\hbar} \eexp{i{\hat{H} t}/{\hbar}}\hat{A}_S \eexp{-i{\hat{H}t}/{\hbar}}\hat{H}+\partial_t\hat{A}_H\\
&= \frac{i}{\hbar}\left[\hat{H},\hat{A}_H\right] + \eexp{i{\hat{H}t}/{\hbar}}\partial_t\hat{A}_S\eexp{-i{\hat{H}t}/{\hbar}}
 \end{align}

In the Heisenberg picture the state vectors are time-in\-de\-pen\-dent:
\begin{align}
	\ket{\psi}_H \equiv \ket{\psi(t=0)}=\eexp{i{\hat{H}}t/{\hbar}} \ket{\psi(t)}.
\end{align}
Therefore, the results of measurements are the same in both pictures:
\begin{align}
\bra{\psi(t)}\hat{A}\ket{\psi(t)} = \bra{\psi}_H \hat{A}_H \ket{\psi}_H.
\end{align}
For example, applying this to the spin operators yields:
\begin{align}						\frac{d}{dt}\hat{s}_{i,H}=\frac{i}{\hbar}\left[\hat{H},\hat{s}_{i,H}\right].
\end{align}