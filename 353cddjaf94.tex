\section{A few words on the quantum information notation}

The qubit is THE basic ingredient of quantum computers. A nice way to play around with them is actually the \href{https://quantum-computing.ibm.com/}{IBM Quantum experience}. However, you will typically not find Pauli matrices etc within these systems. The typical notation there is:
\begin{itemize}
\item $R_x(\phi)$ is a rotation around the x-axis for an angle $\phi$.
\item Same holds for $R_y$ and $R_z$.
\item $X$ denotes the rotation around the x axis for an angle $\pi$. So it transforms $\ket{1}$ into  $\ket{0}$ and vise versa.
\item $Z$ denotes the rotation around the x axis for an angle $\pi$. So it transforms $\ket{+}$ into  $\ket{-}$ and vise versa.
\end{itemize}
The most commonly used gate is actually one that we did not talk about at all, it is the \textit{Hadamard} gate, which transforms $\ket{1}$ into  $\ket{-}$ and $\ket{0}$ into  $\ket{+}$:
\begin{align}
\hat{H}
\end{align}