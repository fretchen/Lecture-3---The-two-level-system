
Such an arbitrary state then reads in general:
\begin{align}
 \ket{\phi} = \left( \begin{array}{c} c_1 \\ c_2 \end{array} \right) = \left( \begin{array}{c} \braket{\varphi_1|\phi} \\ \braket{\varphi_2|\phi} \end{array} \right)
\end{align}
and it can be expressed in terms of the eigenstates of the Hamiltonian:
\begin{align}
	\ket{\phi} = \sum_i c_i \ket{\varphi_i}.
\end{align}
If we apply $\hat{H}_0$ to the arbitrary state, we get:

			\begin{align}
				\ket{\phi'} &=\hat{H}_0 \ket{\phi}\\ 
				&= \sum\limits_{i,j} \ket{\varphi_i}\bra{\varphi_i} \hat{H}_0 \ket{\varphi_j}\braket{\varphi_j|\phi}\\
				&= \sum\limits_i \ket{\varphi_i} \overbrace{\sum\limits_j  \underbrace{\bra{\varphi_i} \hat{H}_0 \ket{\varphi_j} }_{H_{ij}}  \braket{\varphi_j|\phi}}^{c_i'}\\
				&= \sum_i c_i' \ket{\varphi_i}
			\end{align}
			% &= \sum\limits_i \ket{\varphi_i}  \sum\limits_j  \bra{\varphi_i} \hat{H}_0 \ket{\varphi_j}  \braket{\varphi_j|\phi}\\
			% &= \sum\limits_i \ket{\varphi_i} \sum\limits_j  \underbrace{\bra{\varphi_i} \hat{H}_0 \ket{\varphi_j} }_{H_{ij}}  \braket{\varphi_j|\phi}\\
			% &= \sum\limits_i \ket{\varphi_i} \overunderbraces{&&\br{1}{H_{ij}}}{& {\sum\limits_j} & \bra{\varphi_i} \hat{H}_0 \ket{\varphi_j} & \braket{\varphi_j|\phi}}{&\br{3}{c_i'}}\\
			We can also use matrix notation to express $\ket{\phi'}$:
			\begin{align}
				\left( \begin{array}{cc} \bra{\varphi_1}\hat{H}_0\ket{\varphi_1} & \bra{\varphi_1}\hat{H}_0\ket{\varphi_2} \\ \bra{\varphi_2}\hat{H}_0\ket{\varphi_1} & \bra{\varphi_2}\hat{H}_0\ket{\varphi_2} \end{array} \right)\left( \begin{array}{c} \braket{\varphi_1 | \phi} \\ \braket{\varphi_2|\phi} \end{array} \right).
			\end{align}
		\subsection{External Perturbation $\hat{W}$}

			We obtain the perturbed Hamiltonian $\hat{H}$ by adding a small, external perturbation  $\hat{W}$ to the unperturbed Hamiltonian:
			\begin{align} \label{eq:perturbedhamiltonian}
				\hat{H} = \hat{H}_0 + \hat{W}.
			\end{align}
			The eigenstates of the perturbed Hamiltonian will then be:
			\begin{align}
				\hat{H}\ket{\psi_{\pm}}=E_\pm \ket{\psi_\pm}.
			\end{align}

		\subsection{Static Properties}

			In the unperturbed basis $\left\{\ket{\varphi_1},\ket{\varphi_2}\right\}$ we get

			\begin{align}
				\hat{H} = \left(\begin{array}{cc} \tilde{E_1} & W_{12} \\ W^*_{12} & \tilde{E_2}\end{array}\right) \quad \text{with} \quad \tilde{E_i} = E_i + W_{ii}
			\end{align}
			for the perturbed Hamiltonian. Diagonalizing this Hermitian matrix yields
			\begin{align}\label{eq:Epm}
				E_\pm = \frac{1}{2}\left(\tilde{E_1}+\tilde{E_2}\right) \pm \frac{1}{2} \sqrt{\left(\tilde{E}_1-\tilde{E}_2\right)^2+4 \left|W_{12}\right|^2}
			\end{align}
			for the eigenenergies. The corresponding eigenvectors are:
			\begin{align}
				\ket{\psi_+}&=&\hspace{-2mm}&\cos\left(\frac{\theta}{2}\right) \eexp{-i{\varphi}/{2}}\ket{\varphi_1}+\sin\left(\frac{\theta}{2}\right) \eexp{i{\varphi}/{2}}\ket{\varphi_2}, \label{eq:staticpsiplus} \\ 
				\ket{\psi_-}&=&\hspace{-2mm}-&\sin\left(\frac{\theta}{2}\right) \eexp{-i{\varphi}/{2}}\ket{\varphi_1}+\cos\left(\frac{\theta}{2}\right) \eexp{i{\varphi}/{2}}\ket{\varphi_2}, \label{eq:staticpsiminus}
			\end{align}
			where 
			\begin{align} \label{eq:parameters}
				\tan(\theta) = \frac{2|W_{12}|}{\tilde{E}_1-\tilde{E}_2} \quad \text{and} \quad W_{12} \equiv |W_{12}|\cdot \eexp{-i\varphi}.
			\end{align}
			For the further discussion we define:
\begin{align}
	E_m \equiv \frac{1}{2}\left(\tilde{E}_1+\tilde{E}_2\right),\\
	\Delta \equiv \frac{1}{2}\left(\tilde{E}_1-\tilde{E}_2\right),
\end{align}
so that \eqref{eq:Epm} becomes:
\begin{align}
	E_\pm = E_m \pm \sqrt{\Delta^2+|W_{12}|^2}.
\end{align}
