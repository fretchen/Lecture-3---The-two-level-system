\section{Dynamical Aspects}
\subsection{Time Evolution of $\ket{\psi(t)}$}
 After the static case we now want to investigate the dynamical properties of the two-state system. We calculate the time evolution of $\ket{\psi(t)} = c_0(t)\ket{0} + c_1(t)\ket{1}$ with the Schrödinger equation and the perturbed Hamiltonian \eqref{Eq:TwoLevelGeneral}:
\begin{align}
i\hbar \frac{d}{dt}\ket{\psi(t)}&=\hat{H}\ket{\psi(t)},\\
i \frac{d}{dt}\left(\begin{array}{c} c_0(t) \\ c_1(t) \end{array}\right) &= \frac{1}{2}\left( \begin{array}{cc} \Delta & \Omega \\ \Omega^* & -\Delta \end{array} \right) \left(\begin{array}{c} c_0(t) \\ c_1(t) \end{array} \right).
\end{align}

We have two coupled differential equations and we luckily already know how to solve them as we have calculated the two eigenenergies in the previous section. For the state $\ket{\psi(t)}$ we get
\begin{align}
 \ket{\psi(t)}=\lambda \eexp{-i{E_+}t/{\hbar}} \ket{\psi_+} + \mu \eexp{-i{E_-}t/{\hbar}} \ket{\psi_-} \label{eq:psitimeevolution}
\end{align}
with the factors $\lambda$ and $\mu$, which are defined by the initial state. The most common question is then what happens to the system if we start out in the bare state $\ket{0}$ and then let it evolve under coupling with a laser ? So what is the probability to find it in the other state $\ket{1}$:
\begin{align}
	P_1(t)=\left|\braket{1|\psi(t)}\right|^2.
\end{align}
 As a first step, we have to apply the initial condition to \eqref{eq:psitimeevolution} and express $\ket{\varphi}$ in terms of \eqref{eq:staticpsiplus} and \eqref{eq:staticpsiminus}:
\begin{align}
					\ket{\psi(0)} \overset{!}{=}& \ket{0}\\
											  = & \eexp{i{\varphi}/{2}} \left[ \cos\left( \frac{\theta}{2}\right) \ket{\psi_+}-\sin\left(\frac{\theta}{2}\right)\ket{\psi_-}\right]
				\end{align}
				By equating the coefficients we get for $\lambda$ and $\mu$:
				\begin{align}
					\lambda = \eexp{i{\varphi}/{2}}\cos\left(\frac{\theta}{2}\right), \qquad  \mu = -\eexp{i{\varphi}/{2}}\sin\left(\frac{\theta}{2}\right).
				\end{align}
One thus gets:
\begin{align}
\hspace{-2mm} P_1(t)	=&\left|\braket{1|\psi(t)}\right|^2 \\
											=& \left|\eexp{i\varphi} \sin\left(\frac{\theta}{2}\right)\cos\left(\frac{\theta}{2}\right)\left[\eexp{-i{E_+}t/{\hbar}} - \eexp{-i{E_-}t/{\hbar}}\right]\right|^2\\
											=& \sin^2(\theta)\sin^2\left(\frac{E_+-E_-}{2\hbar}t\right)
				\end{align}
$P_1(t)$ can be expressed with $\Delta$ and $\Omega$ alone. The obtained relation is called Rabi's formula:
\begin{align}
 P_1(t)=\frac{1}{1+\left(\frac{\Delta}{|\Omega|}\right)^2}\sin^2\left(\sqrt{|\Omega|^2+\Delta^2}\frac{t}{2}\right)
\end{align}