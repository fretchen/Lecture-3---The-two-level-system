\section{Hamiltonian, Eigenstates and Matrix Notation}

To start out, we will consider two eigenstates $\ket{\varphi_1}$, $\ket{\varphi_2}$ of the Hamiltonian $\hat{H}_0$ with
\begin{align}
 \hat{H}_0\ket{\varphi_1}=E_1\ket{\varphi_1}, \qquad \hat{H}_0\ket{\varphi_2}=E_2\ket{\varphi_2}.
\end{align}
Quite typically we might think of it as a two-level atom with states 1 and 2. The eigenstates can be expressed in matrix notation:
\begin{align}
 \ket{\varphi_1}=\left( \begin{array}{c} 1 \\ 0 \end{array} \right), \qquad \ket{\varphi_2}=\left( \begin{array}{c} 0 \\ 1 \end{array} \right),
\end{align}
so that $\hat{H}_0$ be written as a diagonal matrix
\begin{align}
    \hat{H}_0 = \left(\begin{array}{cc} E_1 & 0 \\ 0 & E_2 \end{array}\right).
\end{align}
If we would only prepare eigenstates the system would be rather boring. However, we typically have the ability to change the Hamiltonian by switching on and off laser or microwave fields. \footnote{See the discussions of the next lecture}. We can then write the Hamiltonian in its most general form as:
\begin{align}
\hat{H} = \frac{\hbar}{2}\left( \begin{array}{cc} \Delta  & \Omega_x - i\Omega_y\\ \Omega_x +i\Omega_y & -\Delta \end{array} \right)
\end{align}

\subsection{Case of no perturbation $\Omega_x = \Omega_y = 0$}

This is exactly the case of no applied laser fields that we discussed previously. We simply removed the energy offset $E_m = \frac{E_1+E_2}{2}$ and pulled out the factor $\hbar$, such that $\Delta$ measures a frequency. So we have:
\begin{align}
E_1 = E_m+ \frac{\hbar}{2}\Delta\\
E_2 = E_m- \frac{\hbar}{2}\Delta
\end{align}
We typically call $\Delta$ the energy difference between the levels or the \textbf{detuning}.

\subsection{Case of no detuning $\Delta = 0$}

Let us suppose that the diagonal elements are exactly zero. And for simplicity we will also keep $\Omega_y =0$ as it simply complicates the calculations without adding much to the discussion at this stage. The Hamiltonian reads then:
\begin{align}
\hat{H} = \frac{\hbar}{2}\left( \begin{array}{cc} 0  & \Omega_x\\ \Omega_x &0 \end{array} \right)
\end{align}

Quite clearly the st