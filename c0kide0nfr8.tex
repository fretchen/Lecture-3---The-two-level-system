\section{Hamiltonian, Eigenstates and Matrix Notation}

To start out, we will consider two eigenstates $\ket{\varphi_1}$, $\ket{\varphi_2}$ of the Hamiltonian $\hat{H}_0$ with
\begin{align}
 \hat{H}_0\ket{\varphi_1}=E_1\ket{\varphi_1}, \qquad \hat{H}_0\ket{\varphi_2}=E_2\ket{\varphi_2}.
\end{align}
Quite typically we might think of it as a two-level atom with states 1 and 2. The eigenstates can be expressed in matrix notation:
\begin{align}
 \ket{\varphi_1}=\left( \begin{array}{c} 1 \\ 0 \end{array} \right), \qquad \ket{\varphi_2}=\left( \begin{array}{c} 0 \\ 1 \end{array} \right),
\end{align}
so that $\hat{H}_0$ be written as a diagonal matrix
\begin{align}
    \hat{H}_0 = \left(\begin{array}{cc} E_1 & 0 \\ 0 & E_2 \end{array}\right).
\end{align}
This would b
If we would only prepare eigenstates the system would be rather boring. However, we typically have the ability to change the Hamiltonian by switching on and off laser or microwave fields. \footnote{See the discussions of the next lecture}. We can then write the Hamiltonian in its most general form as:
\begin{align}
H = \frac{\hbar}{2}\left( \begin{array}{cc} \Delta  & \Omega_x - i\Omega_y\\ \Omega_x +i\Omega_y & -\Delta \end{array} \right)
\end{align}

Such an arbitrary state then reads in general:
\begin{align}
 \ket{\phi} = \left( \begin{array}{c} c_1 \\ c_2 \end{array} \right) = \left( \begin{array}{c} \braket{\varphi_1|\phi} \\ \braket{\varphi_2|\phi} \end{array} \right)
\end{align}
and it can be expressed in terms of the eigenstates of the Hamiltonian:
\begin{align}
	\ket{\phi} = \sum_i c_i \ket{\varphi_i}.
\end{align}
If we apply $\hat{H}_0$ to the arbitrary state, we get:

			\begin{align}
				\ket{\phi'} &=\hat{H}_0 \ket{\phi}\\ 
				&= \sum\limits_{i,j} \ket{\varphi_i}\bra{\varphi_i} \hat{H}_0 \ket{\varphi_j}\braket{\varphi_j|\phi}\\
				&= \sum\limits_i \ket{\varphi_i} \overbrace{\sum\limits_j  \underbrace{\bra{\varphi_i} \hat{H}_0 \ket{\varphi_j} }_{H_{ij}}  \braket{\varphi_j|\phi}}^{c_i'}\\
				&= \sum_i c_i' \ket{\varphi_i}
			\end{align}
			% &= \sum\limits_i \ket{\varphi_i}  \sum\limits_j  \bra{\varphi_i} \hat{H}_0 \ket{\varphi_j}  \braket{\varphi_j|\phi}\\
			% &= \sum\limits_i \ket{\varphi_i} \sum\limits_j  \underbrace{\bra{\varphi_i} \hat{H}_0 \ket{\varphi_j} }_{H_{ij}}  \braket{\varphi_j|\phi}\\
			% &= \sum\limits_i \ket{\varphi_i} \overunderbraces{&&\br{1}{H_{ij}}}{& {\sum\limits_j} & \bra{\varphi_i} \hat{H}_0 \ket{\varphi_j} & \braket{\varphi_j|\phi}}{&\br{3}{c_i'}}\\
			We can also use matrix notation to express $\ket{\phi'}$:
			\begin{align}
				\left( \begin{array}{cc} \bra{\varphi_1}\hat{H}_0\ket{\varphi_1} & \bra{\varphi_1}\hat{H}_0\ket{\varphi_2} \\ \bra{\varphi_2}\hat{H}_0\ket{\varphi_1} & \bra{\varphi_2}\hat{H}_0\ket{\varphi_2} \end{array} \right)\left( \begin{array}{c} \braket{\varphi_1 | \phi} \\ \braket{\varphi_2|\phi} \end{array} \right).
			\end{align}

