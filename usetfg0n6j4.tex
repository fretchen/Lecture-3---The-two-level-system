\subsection{Side notes concerning the spin picture}

While the previous derivation might be the standard one, which certainly leads to the right results it might not be the most intuitive way of thinking about the dynamics. They become actually quite transparent in the spin language and on the Bloch sphere. So let us go back to the formulation of the Hamiltonian in terms of spins as in Eq. \eqref{Eq:HamSpin}.

How would the question of the time evolution from $0$ to $1$ and back go now ? Basically, we would assume that the spin has been initialize into one of the eigenstates of the $z$-basis and now starts to rotate in some magnetic field. How ? This can be nicely studied in the Heisenberg picture, where operators have a time evolution. In the Heisenberg picture we have:
\begin{align}
\frac{d}{dt} \hat{s}_z &= \frac{i}{\hbar}\left[\hat{H},\hat{s}_H\right] + \eexp{i{\hat{H}t}/{\hbar}}\partial_t\hat{A}_S\eexp{-i{\hat{H}t}/{\hbar}}
 \end{align}

In the Heisenberg picture the state vectors are time-in\-de\-pen\-dent:
\begin{align}
	\ket{\psi}_H \equiv \ket{\psi(t=0)}=\eexp{i{\hat{H}}t/{\hbar}} \ket{\psi(t)}.
\end{align}
Therefore, the results of measurements are the same in both pictures:
\begin{align}
\bra{\psi(t)}\hat{A}\ket{\psi(t)} = \bra{\psi}_H \hat{A}_H \ket{\psi}_H.
\end{align}
For example, applying this to the spin operators yields:
\begin{align}						\frac{d}{dt}\hat{s}_{i,H}=\frac{i}{\hbar}\left[\hat{H},\hat{s}_{i,H}\right].
\end{align}